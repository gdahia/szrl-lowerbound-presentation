\documentclass{beamer}

\usetheme{Rochester}

\usepackage{amsthm,amsmath,amssymb,bbm}

\DeclareMathOperator{\twr}{\mathrm{tower}}
\newcommand{\1}{\mathbbm{1}}
\newcommand{\indicator}[1]{\1_{[#1]}}
\newcommand{\Exp}[1]{\mathbb{E}\left [#1 \right ]}
\newcommand{\Prob}{\mathbb{P}}
\newcommand{\V}[1]{{V \choose #1}}
\newcommand{\Z}{\mathcal{Z}}
\newcommand{\eps}{\varepsilon}

\title{Lower bounds for Szemer\'{e}di's Regularity Lemma}

\subtitle{By Gowers (1997), Moshkovitz and Shapira (2016)}

\author{Gabriel Dahia}

\institute{IMPA}

\date{\today}

\begin{document}

\beamertemplatenavigationsymbolsempty

\begin{frame}
  %
  \titlepage
  %
\end{frame}

\begin{frame}{Based on}
  %
  \begin{itemize}
    %
    \item Gowers, W.T. ``\textit{Lower bounds of tower type for Szemer\'{e}di's
          uniformity lemma}.'' Geometric \& Functional Analysis (1997).
          %
    \item Moshkovitz, G. and Shapira, A. ``\textit{A short proof of Gowers’ lower bound
            for the regularity lemma}.'' Combinatorica (2016).
          %
          %\item Fox, J., Lov\'{a}sz, L.M. ``\textit{A tight lower bound for
          %Szemer\'{e}di’s regularity lemma}.'' Combinatorica (2017).
          %
  \end{itemize}
  %
\end{frame}

\begin{frame}{Definitions}
  %
  \begin{definition}[$\eps$-regular pair $(A, B)$]
    %
    $|d(A, B) - d(X, Y)| \le \eps$ for all $A \subseteq X$ and $B \subseteq Y$ if
    $|A| \ge \eps |X|$ and $|B| \ge \eps |Y|$.
    %
  \end{definition}

  \pause
  %
  \begin{definition}[Equipartition $\Z$]
    %
    $\Z = \{Z_1, \dots, Z_k\}$ partition such that $|Z_i - Z_j| \le 1$ for all $i, j \in
      [k]$.
    %
  \end{definition}

  \pause
  %
  \begin{definition}[$\eps$-regular equipartition $\Z$]
    %
    All but $\eps k^2$ pairs $(Z_i, Z_j)$ are $\eps$-regular.
    %
  \end{definition}
  %
\end{frame}

\begin{frame}{Szemer\'{e}di's Regularity Lemma}
  %
  \begin{theorem}[Szemer\'{e}di (1978)]
    %
    For every $0 < \eps < 1/2$, there is $M = M(\eps)$ so that every graph has an
    $\eps$-regular equipartition with at most $M$ parts.
    %
    \pause
    %
    Moreover, $M(\eps) \le \twr(O(\eps^{-1/5}))$.
    %
  \end{theorem}
  %
\end{frame}

\begin{frame}{Lowerbounds for $M(\eps)$}
  %
  \begin{block}{Question}
    %
    For which functions $f(\eps)$ do we have a graph such that $f(\eps) \le M(\eps)$?
    %
  \end{block}
  %
  % TODO: motivate this question: better bounds for applications of regularity, Gowers'
  % remark that it would be more useful if it were of the form exp(eps^-beta).

  %
  \pause
  %
  \begin{theorem}[Gowers (1997)]
    %
    There exists a graph and a constant $c > 0$ such that $\twr(\eps^{-c}) \le M(\eps)$.
    %
  \end{theorem}

  \pause

  \begin{corollary}[Gowers (1997)]
    %
    There exists a graph and a constant $c > 0$ such that $\twr(-c\log \eps) \le
      M(\eps)$.
    %
  \end{corollary}
  %
  % TODO: note that gowers prove his lowerbound for a weaker version of the szrl, so the
  % proof is more general. note however that the ideas are the same
  %
\end{frame}

%\begin{frame}{Alternative equivalent setting}
%%
%\begin{definition}[$\eps$-nice equipartition]
%%
%For all $Z \in \Z$, all but $k \eps$ sets $Z' \in \Z$ are such that $(Z, Z')$ is
%$\eps$-regular.
%%
%\end{definition}

%\pause

%\begin{block}{Claim}
%%
%Let $M'(\eps)$ be so that every graph has a $\eps$-nice equipartition. Then,
%$M'(\eps) \le M(\eps^3)^2$.
%%
%\end{block}

%\pause

%\begin{proof}
%%
%%
%\end{proof}
%%
%\end{frame}

% TODO: add fox and lovasz overview of the proof technique, including how the weights are
% determined. try to use that to motivate the definitions.

\begin{frame}{Gowers lowerbound framework}
  %
  Use edge distribution instead of explicit graph.
  %
  \pause
  %
  Define a sequence of equipartitions $\Z_0, \dots, \Z_s$ with $\Z_{i + 1}$ refining
  $\Z_i$ in exponentially more parts.
  %
  \pause
  %
  Probability of edge $uv \in G$ depends on each parts both $u$ and $v$ lie in.
  %
  \pause
  %
  Any $\eps$-regular partition of the edge distribution cannot be too far from being a
  refinement of $\Z_s$, which has many parts.
  %
\end{frame}

\begin{frame}{Alternative, equivalent setting}
  %
  \begin{definition}[Pair density for edge distributions]
    %
    $d_\mu(A, B) = |A|^{-1} |B|^{-1} \sum_{(u, v) \in A \times B} \Prob_\mu(uv \in G)$.
    %
  \end{definition}

  \pause

  \begin{lemma}
    %
    If $G$ is sampled from $\mu$, then for all $m \gtrsim \theta^{-2} \log n$, $|d_\mu(A,
      B) - d(A, B)| \le \theta$ with probability at least 1/2 if $|A| = |B| = m$.
    %
  \end{lemma}

  \pause

  \begin{proof}
    %
    \pause
    %
    If $G$ is sampled from $\Prob$, then $d(A, B) = |A|^{-1} |B|^{-1} \sum_{uv}
      \indicator{uv \in G}$.
    %
    \pause
    %
    Then, $\Exp{d(A, B)} = d_\mu(A, B)$.
    %
    \pause
    %
    By Chernoff's bound, $\Prob(|d(A, B) - d_\mu(A, B)| \ge \theta) \le 2\exp(-2
      m^2 \theta^2)$.
    %
    \pause
    %
    Taking a union bound over the choices of $A$ and $B$, the probability of $G$
    failing the lemma is $\Prob(G \text{ bad}) \le 2 {n \choose m}^2 \exp(-2 m^2
      \theta^2)$,
    %
    \pause
    %
    which is less than 1/2 if $m \ge C \theta^{-2} \log n$ for some constant $C$.
    %
  \end{proof}
  %
\end{frame}

\begin{frame}{$c$-balanced partitions}
  %
  \begin{definition}[$c$-balanced bipartitions]
    %
    Sequence $(A_i, B_i)_{i = 1}^m$ of equal-sized-parts bipartitions of $M$ such that
    for $s, t \in [M]$ there are at most $(1/2 + c)m$ values of $i$ for which $s$ and
    $t$ lie in the same part of the corresponding bipartition.
    %
  \end{definition}

  \pause

  \begin{lemma}
    %
    For every $m \ge 1$ and $M \lesssim 2^{c^2 m}$, there exists a sequence of
    $c$-balanced bipartitions of $[M]$.
    %
  \end{lemma}

  \pause

  \begin{proof}
    %
    \pause
    %
    For bipartition $i$, and element $t \in [M]$, put $t \in A_i$ with probability 1/2
    and in $B_i$ otherwise.
    %
    \pause
    %
    Let $X_{s, t}$ be the random variable counting the times $s, t$ fall in the same
    side.
    %
    \pause
    %
    $\Exp{X_{s, t}} = m/2$. Using Chernoff's bound, $\Prob(X_{s, t} > (1 + c)m/2) \le
      \exp(-c^2 m/4)$.
    %
    \pause
    %
    Take a union bound over all ${M \choose 2}$ choices of distinct $s, t$.
    %
  \end{proof}
  %
\end{frame}

\end{document}
