\documentclass{beamer}

\usetheme{Rochester}

\DeclareMathOperator{\Hy}{\mathcal{H}}
\DeclareMathOperator{\Prob}{\mathbb{P}}
\DeclareMathOperator{\E}{\mathbb{E}}
\DeclareMathOperator{\m}{\mathbf{m}}
\DeclareMathOperator{\q}{\mathbf{q}}
\DeclareMathOperator{\eps}{\varepsilon}
\newcommand{\V}[1]{{V \choose #1}}

\title{EKR for random hypergraphs}

\subtitle{Based on ``On Erd\H{o}s-Ko-Rado for random hypergraphs I'', Arran Hamm, Jeff
  Kahn, 2018}

\author{Gabriel Dahia}

\institute{IMPA}

\date{\today}

\begin{document}

\beamertemplatenavigationsymbolsempty

\begin{frame}
  %
  \titlepage
  %
\end{frame}

\begin{frame}{A brief recap of EKR}
  %
  \begin{definition}[$k$-uniform hypergraph]<1->
    %
    $\Hy \subseteq \V{k}$, $|V| = n > 2k$.
    %
  \end{definition}

  \begin{definition}[Clique / intersecting family]<2->
    %
    $A \cap B \neq \emptyset$ for $A, B \in \Hy$.
    %
  \end{definition}

  \begin{definition}[Star]<3->
    %
    $\Hy_x = \{A \in \Hy \mid x \in A\}$
    %
  \end{definition}

  \begin{theorem}[Erd\H{o}s, Ko, Rado; 1961]<4->
    %
    For $\Hy$ clique:
    %
    \begin{enumerate}
      %
      \item $|\Hy| \le {n - 1 \choose k - 1}$.
            %
      \item<5-> $|\Hy| = {n - 1 \choose k - 1}$ iff $\Hy =
              \V{k}_x$ for some $x \in V$.
            %
    \end{enumerate}
    %
  \end{theorem}

\end{frame}

\begin{frame}{Erd\H{o}s-Ko-Rado property}

  \begin{definition}<1->
    %
    $\Hy$ \emph{is EKR} if every largest clique of $\Hy$ is a star.
    %
  \end{definition}

  \begin{theorem}[Erd\H{o}s, Ko, Rado; 1961]<2->
    %
    \begin{enumerate}
      %
      \setcounter{enumi}{1}
      %
      \item $\V{k}$ is EKR.
            %
    \end{enumerate}
    %
  \end{theorem}

\end{frame}

\begin{frame}{EKR for random $\Hy$}
  %
  \begin{definition}[$\Hy_k(n, p)$]
    %
    Random $\Hy$, elements of $\V{k}$ independently with probability $p$.
    %
    % repeated elements?
    %
  \end{definition}
  %
  % this is our G(n, p)
  %
  \begin{block}{Question (Balogh, Bohman, Mubayi; 2009)}<2->
    %
    ``For what $p_0 = p_0(n,k)$ is it true that $\Hy_k(n, p)$ satisfies EKR almost surely
    provided $p > p_0$?''
    %
    % similar to questions of ex(n, H) generalize to ex(G(n, p), H). ie V choose k is K_n
    %
  \end{block}
  %
  \begin{block}{Remark}<3->
    %
    EKR not increasing by edge addition, nor by increasing $p$.
    %
  \end{block}
  %
\end{frame}

\begin{frame}{Notation}
  %
  \begin{definition}[$\varphi$, expected degree of a vertex]
    %
    $\varphi = p {n - 1 \choose k - 1}$
    %
  \end{definition}
  %
  \begin{definition}[$m$, expected number of edges in $\Hy$]<2->
    %
    $m = \E[|\Hy|] = \varphi n/k$
    %
  \end{definition}
  %
  \begin{definition}[$q$, intersection probability]<3->
    %
    $q = \Prob (A \cap B \neq \emptyset)$, $A, B$, indep., uniformly at random.
    %
  \end{definition}
  %
  \begin{definition}[$\Lambda$, expected number of ``generic'' cliques]<4->
    %
    $\Lambda(t) = {m \choose t} q^{t \choose 2}$
    %
    % \Lambda(t) equals the expected number of generic t-cliques: the first {m \choose t}
    % is the number of choices of a t-subset, and q^{t \choose 2} is the probability of
    % it being a clique
    %
  \end{definition}
  %
\end{frame}

\begin{frame}{$q$ and EKR}
  %
  \begin{definition}[$q$, intersection probability]
    %
    $q = \Prob (A \cap B \neq \emptyset)$, $A, B$, indep., uniformly at random.
    %
  \end{definition}
  %
  \begin{block}{$k \ll n^{1/2}$}
    %
    \begin{enumerate}
      %
      \item<2-> $q \to 0$
            %
      \item<3-> $A$ and $B$ typically disjoint
            %
      \item<4-> main focus of Balogh, Bohman, Mubayi (2009).
            %
    \end{enumerate}
    %
  \end{block}
  %
  \begin{block}{$k = \Omega(n^{1/2})$}<5->
    %
    \begin{enumerate}
      %
      \item<6-> $A$ and $B$ typically intersect
            %
      \item<7-> main focus of Hamm, Kahn (2018)
            %
    \end{enumerate}
    %
  \end{block}
  %
\end{frame}

\begin{frame}{What we are going to see today}
  %
  \begin{theorem}[1.1 in Hamm, Kahn (2018)]
    %
    For any fixed $c < 1/4$, if
    %
    \begin{enumerate}
      %
      \item<2-> $k < \sqrt{c n \log n}$
            %
      \item<3-> $\varphi$ is such that $\Prob(\Lambda(\Delta_{\Hy}) > \eps) \to 0$,
            %
    \end{enumerate}
    %
    \onslide<4->
    %
    then, $\Prob(\Hy \vDash \mathrm{EKR}) \to 1$.
    %
    % explain why we should not expect c > 1/2
    %
    % explain why we should not expect it to hold if \Lambda is large: the q term is only
    % the real probability when the events are independent. this is not what happens for
    % a star, and so the number of non-star maximum cliques should be vanishing
    %
  \end{theorem}
  %
  \begin{block}{Not quite}<5->
    %
    We will assume throught $k > n^{1/2 - o(1)}$ and $c > 1/4 - \eps$.
    %
  \end{block}
  %
\end{frame}

% observation: EKR can be written as saying the maximum clique is of size equal to the
% maximum degree and it is trivial. lemmas 2.1-2.3 crucially rely on that

\end{document}
