\documentclass{beamer}

\usetheme{Rochester}

\DeclareMathOperator{\Hy}{\mathcal{H}}
\DeclareMathOperator{\Prob}{\mathbb{P}}
\newcommand{\V}[1]{{V \choose #1}}

\title{EKR for random hypergraphs}

\subtitle{Based on ``On Erd\H{o}s-Ko-Rado for random hypergraphs I'', Arran Hamm, Jeff
  Kahn, 2018}

\author{Gabriel Dahia}

\institute{IMPA}

\date{\today}

\begin{document}

\begin{frame}
  %
  \titlepage
  %
\end{frame}

\begin{frame}{A brief recap of EKR}
  %
  \begin{definition}[$k$-uniform hypergraph]<1->
    %
    $\Hy \subseteq \V{k}$, $|V| = n > 2k$.
    %
  \end{definition}

  \begin{definition}[Clique / intersecting family]<2->
    %
    $A \cap B \neq \emptyset$ for $A, B \in \Hy$.
    %
  \end{definition}

  \begin{definition}[Star]<3->
    %
    $\Hy_x = \{A \in \Hy \mid x \in A\}$
    %
  \end{definition}

  \begin{theorem}[Erd\H{o}s, Ko, Rado; 1961]<4->
    %
    For $\Hy$ clique:
    %
    \begin{enumerate}
      %
      \item $|\Hy| \le {n - 1 \choose k - 1}$.
            %
      \item<5-> $|\Hy| = {n - 1 \choose k - 1}$ iff $\Hy =
              \V{k}_x$ for some $x \in V$.
            %
    \end{enumerate}
    %
  \end{theorem}

\end{frame}

\begin{frame}{Erd\H{o}s-Ko-Rado property}

  \begin{definition}<1->
    %
    $\Hy$ \emph{is EKR} if every largest clique of $\Hy$ is a star.
    %
  \end{definition}

  \begin{theorem}[Erd\H{o}s, Ko, Rado; 1961]<2->
    %
    \begin{enumerate}
      %
      \setcounter{enumi}{1}
      %
      \item $\V{k}$ is EKR.
            %
    \end{enumerate}
    %
  \end{theorem}

\end{frame}

\begin{frame}{EKR for random $\Hy$}
  %
  \begin{definition}[$\Hy_k(n, p)$]
    %
    Random $\Hy$, elements of $\V{k}$ independently with probability $p$.
    %
    % repeated elements?
    %
  \end{definition}
  %
  % this is our G(n, p)
  %
  \begin{block}{Question (Balogh, Bohman, Mubayi; 2009)}<2->
    %
    ``For what $p_0 = p_0(n,k)$ is it true that $\Hy_k(n, p)$ satisfies EKR almost surely
    provided $p > p_0$?''
    %
    % similar to questions of ex(n, H) generalize to ex(G(n, p), H). ie V choose k is K_n
    %
  \end{block}
  %
  \begin{block}{Remark}<3->
    %
    EKR not increasing by edge addition, nor by increasing $p$.
    %
  \end{block}
  %
\end{frame}

\begin{frame}{What was known before HM2018}
  %
  \begin{definition}[Expected degree of a vertex]
    %
    $\varphi = p {n - 1 \choose k - 1}$
    %
  \end{definition}
  %
  \begin{theorem}[Balogh, Bohman, Mubayi; 2009]
    %
    \begin{enumerate}
      %
      \item If $k \ll n^{1/4}$, then $\Hy \vDash \textrm{EKR}$.
            %
      \item If $n^{1/4} \ll k \ll n^{1/3}$ and $\varphi \ll k^{-1}$ or $n^{-1/4} \ll
              \varphi$, then $\Hy \vDash \textrm{EKR}$.
            %
    \end{enumerate}
    %
  \end{theorem}
  %
\end{frame}

% \Lambda(t) equals the expected number of generic t-cliques: the first {m \choose t} is
% the number of choices of a t-subset, and q^{t \choose 2} is the probability of it being
% a clique

\end{document}
